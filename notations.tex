% **********************************************************************
% Copyright 2020-2024 Aleksander GRM

% Author: Aleksander GRM @fpp.uni-lj.si
% Description: This is latex addition file. 
%              Feel free to use it, modify it, share it.
% Version: 1.0
% Date: 02/05/2024
% URL: https://github.com/as-grm/latex_templates
% **********************************************************************


%
% *** Needed packages ***
%
\usepackage{dsfont}

%
% *** New Commands ***
%
%\newcommand*{\point}[1]{\vec{\mkern0mu#1}}
\newcommand{\ci}[0]{\perp\!\!\!\!\!\perp} % conditional independence
\newcommand{\point}[1]{{#1}} % points 
\renewcommand{\vec}[1]{\boldsymbol{#1}}  % math vector
\newcommand{\uvec}[1]{\mathbf{\hat{#1}}} % unit vector
%\newcommand{\mat}[1]{\boldsymbol{#1}}    % matrix
\newcommand{\mat}[1]{\boldsymbol{\mathsf{#1}}}    % matrix
\newcommand{\RR}[0]{\mathbb{R}} % real numbers
\newcommand{\ZZ}[0]{\mathbb{Z}} % integers
\newcommand{\NN}[0]{\mathbb{N}} % natural numbers
\newcommand{\QQ}[0]{\mathbb{Q}} % rational numbers
\newcommand{\CC}[0]{\mathbb{C}} % complex numbers
\newcommand{\tr}[0]{\text{tr}} % trace
\renewcommand{\d}[0]{\mathrm{d}} % total derivative
\newcommand{\prt}[0]{\partial}
\newcommand{\inv}{^{-1}} % inverse
\newcommand{\id}{\mathrm{id}} % identity mapping
\renewcommand{\dim}{\mathrm{dim}} % dimension
%\newcommand{\dim}[1]{\text{\textsc{dim}}\left(#1\right)} 
\newcommand{\adj}[1]{\mathrm{adj}(#1)} % adjoint or adjugate matrix
\newcommand{\rank}[1]{\mathrm{rk}(#1)} % rank
\newcommand{\trace}[1]{\mathrm{sl}(#1)} % trace ali sled
\newcommand{\determ}[1]{\mathrm{det}(#1)} % determinant
%\renewcommand{\det}[1]{\mathrm{det}(#1)} % determinant
\newcommand{\scp}[2]{\langle #1 , #2 \rangle}
\newcommand{\kernel}[1]{\mathrm{ker}({#1})} % kernel/nullspace
\newcommand{\img}[0]{\mathrm{Im}} % image
\newcommand{\idx}[1]{{(#1)}} % index
\newcommand{\diag}[1]{{\mathrm{diag}(#1)}} % diagonal operator
\newcommand{\cov}[1]{{\mathrm{cov}(#1)}} % covariance (matrix) 
\newcommand{\mean}{\mathds{E}} % expectation
\newcommand{\var}{\mathds{V}} % variance
\newcommand{\gauss}[2]{\mathcal{N}\big(#1,\,#2\big)} % gaussian distribution N(.,.)
\newcommand{\gaussx}[3]{\mathcal{N}\big(#1\,|\,#2,\,#3\big)} % gaussian distribution N(.|.,.)
\newcommand{\gaussBig}[2]{\mathcal{N}\left(#1,\,#2\right)} % see above, but with brackets that adjust to the height of the arguments
\newcommand{\gaussxBig}[3]{\mathcal{N}\left(#1\,|\,#2,\,#3\right)} % see above, but with brackets that adjust to the height of the arguments

\newcommand{\T}[0]{^\top} % transpose
\newcommand{\imu}[0]{\mathrm{i}} % complex unit i
\newcommand{\matdet}[1]{\left|\begin{matrix}#1\end{matrix}\right|} % determinant of a matrix
\newcommand{\vecnorm}[1]{\lVert#1\rVert}
\newcommand{\absval}[1]{\left|#1\right|}
\newcommand{\scaprod}[2]{\langle #1, #2 \rangle_0}
\newcommand{\operator}[2]{\operatorname{#1}\left[#2\right]}
\newcommand{\order}[1]{\mathcal{O}(#1)}
\newcommand{\vertat}[2]{\left.#1\:\right|_{#2}}
\newcommand{\longprt}[2]{\frac{\prt #1}{\prt #2}}
\newcommand{\longprtt}[2]{\frac{\prt^2 #1}{\prt #2^2}}


%
% *** Arc notations ***
% Use it only in math mode \[ ... \]
%
\newcommand{\txtdeg}{\si{\degree}}
\newcommand{\txtmin}{\si{\arcmiute}}
\newcommand{\txtsec}{\si{\arcsecond}}
\newcommand{\arcdeg}[1]{\SI{#1}{\degree}} 
\newcommand{\arcmin}[1]{\SI{#1}{\arcminute}} 
\newcommand{\arcsec}[1]{\SI{#1}{\arcsecond}}
\newcommand{\posdm}[2]{\SI{#1}{\degree}\:\SI{#2}{\arcminute}}
\newcommand{\posdmo}[3]{\SI{#1}{\degree}\:\SI{#2}{\arcminute}\:\text{#3}}
\newcommand{\posdms}[3]{\SI{#1}{\degree}\:\SI{#2}{\arcminute}\:\SI{#3}{\arcsecond}}

%
% *** some shotcuts, bolded symbols, fnacy style, ... ***
%
\newcommand{\mc}{\mathcal}
\newcommand{\mb}{\mathbb}
\newcommand{\ul}{\underline}
\newcommand{\tb}{\textbf}
\newcommand{\bk}{\boldkey}
\newcommand{\bs}{\boldsymbol}
\newcommand{\circled}[1]{\tikz[baseline=(char.base)]{\node[shape=circle,draw,inner sep=1pt] (char) {#1};}}

%
% *** Trade names, Copy right... ***
%
% \textsuperscript \textsubscript
\newcommand{\matlab}{\texttt{MatLab}\textsuperscript{\textregistered}~}
\newcommand{\cpp}{\texttt{c++}\textsuperscript{\textregistered}~}
\newcommand{\python}{\texttt{Python}\textsuperscript{\textregistered}~}
\newcommand{\scipy}{\texttt{SciPy}\textsuperscript{\textregistered}~}
\newcommand{\sympy}{\texttt{SymPy}\textsuperscript{\textregistered}~}

%
% *** dimension less numbers, and other numbers ***
%
\newcommand{\RE}{\text{\textsl{Re}}}
\newcommand{\MA}{\text{\textsl{Ma}}}
\newcommand{\KN}{\text{\textsl{Kn}}}
%\newcommand{\half}{\sfrac{1}{2}}
\newcommand{\half}{1/2}


%
% *** tabular ***
%
\newcommand{\centercell}[1]{\multicolumn{1}{c}{#1}}
\newcommand{\head}[1]{\centercell{\bfseries#1}}

%
% *** various color definitions ***
%
\definecolor{darkgreen}{rgb}{0,0.6,0}
\definecolor{royalblue1}{RGB}{72,118,255}
\definecolor{shadowbg}{RGB}{51,51,51}
\definecolor{titlebg}{RGB}{51,51,51}
\definecolor{royalblue4}{RGB}{39,64,139}
\definecolor{navyblue}{RGB}{0, 0, 127}
\definecolor{deepcarminepink}{rgb}{0.94, 0.19, 0.22}
\definecolor{navy}{rgb}{0.177,0.349,0.529}
\definecolor{navylight}{rgb}{0.761,0.828,0.894}
\definecolor{myblue}{rgb}{.9, .9, 1}
\definecolor{lightblue}{rgb}{.9, .9, .5}

%
% *** redefine EMPH types ***
%
\newcommand{\missing}[1]{\textcolor{red}{\textbf{???#1???}}}
\newcommand{\emphc}[1]{\textcolor{blue}{\textbf{#1}}}
\newcommand{\emphcb}[1]{\textcolor{deepcarminepink}{\textbf{#1}}}
\newcommand{\myfig}[1]{\iflanguage{english}{\textbf{Figure}}{\textbf{Slika}} #1}
\newcommand{\myeq}[1]{\iflanguage{english}{\textbf{Eq.}}{\textbf{En.}} (#1)}

%
% *** place a colored box around a character ***
%
\gdef\colchar#1#2{%
	\tikz[baseline]{%
		\node[anchor=base,inner sep=2pt,outer sep=0pt,fill = #2!20] {#1};
	}%
}%


% ***********************************************
% *** Blocks: block, exampleblock, alertblock ***
% ***********************************************

%
% *** Blocks: block, exampleblock, alertblock ***
%
%\newcommand*\mybluebox[1]{%
	%	\colorbox{myblue}{\hspace{1em}#1\hspace{1em}}}

\newlength\mytemplen
\newsavebox\mytempbox

\makeatletter
\newcommand\mybluebox{%
	\@ifnextchar[%]
	{\@mybluebox}%
	{\@mybluebox[0pt]}}

\def\@mybluebox[#1]{%
	\@ifnextchar[%]
	{\@@mybluebox[#1]}%
	{\@@mybluebox[#1][0pt]}}

\def\@@mybluebox[#1][#2]#3{
	\sbox\mytempbox{#3}%
	\mytemplen\ht\mytempbox
	\advance\mytemplen #1\relax
	\ht\mytempbox\mytemplen
	\mytemplen\dp\mytempbox
	\advance\mytemplen #2\relax
	\dp\mytempbox\mytemplen
	\colorbox{myblue}{\hspace{1em}\usebox{\mytempbox}\hspace{1em}}}
\makeatother


% *** Example Boxed Equation ***
%

% For a single line equation use: box={\mybluebox[5pt][-2pt]}

% \begin{empheq}[box=\mybluebox]{equation*}
	%   c(t) = c_{in}\left[ 1 - \left(1+ \frac{\Delta \phi}{V_0} \: t
	%   \right)^{-\frac{\phi_{in}}{\Delta \phi}} \right]
	% \end{empheq}
%
% \begin{empheq}[box={\mybluebox[5pt]}]{equation*}
	% 	c_i = \sum_j A_{ij}
	% \end{empheq}
%
% \begin{empheq}[box={\mybluebox[2pt][2pt]}]{equation*}
	%	c_i = \langle\psi|\phi\rangle
	% \end{empheq}


% *********************************************************
% *** Inline code listings: Matlab, Python pretty print ***
% *********************************************************

\usepackage{listings}
\usepackage{xcolor}

\definecolor{codegreen}{rgb}{0,0.6,0}
\definecolor{codegray}{rgb}{0.5,0.5,0.5}
\definecolor{codepurple}{rgb}{0.58,0,0.82}
\definecolor{backcolour}{rgb}{0.95,0.95,0.92}

\lstdefinestyle{mystyle}{
	backgroundcolor=\color{backcolour},   
	commentstyle=\color{codegreen},
	keywordstyle=\color{magenta},
	numberstyle=\tiny\color{codegray},
	stringstyle=\color{codepurple},
	basicstyle=\ttfamily\footnotesize,
	breakatwhitespace=false,         
	breaklines=true,                 
	captionpos=b,                    
	keepspaces=true,                 
	numbers=left,                    
	numbersep=5pt,                  
	showspaces=false,                
	showstringspaces=false,
	showtabs=false,                  
	tabsize=2
}


% *** Example ***
 
% language: Matlab-editor, Matlab-Pyglike, Python, Octave, C, C++,
% look in help doc for additional languages  

%\begin{frame}[fragile]{Title}
%
%  \lstinputlisting[language=Octave,style=mystyle]{BitXorMatrix.m}
%
%\end{frame}


% ******************************
% *** Definicija slo za math ***
% ******************************
%\newtheorem{izrek}{Izrek}
%\newtheorem{lema}[izrek]{Lema}
%\newtheorem{trditev}[izrek]{Trditev}
%\newtheorem{posledica}[izrek]{Posledica}
%\newtheorem{definicija}[izrek]{Definicija}
%\newtheorem{vaja}[izrek]{Vaja}



% ********************
% !!!!! Examples !!!!!
% ********************

% 1. Minpage with top
%
%\begin{minipage}[t]{7cm}
%	\vspace{0pt}
%\end{minipage}
%\hfill
%\begin{minipage}[t]{7cm}
%	\vspace{0pt}
%\end{minipage}